\documentclass{beamer}

\pdfmapfile{+sansmathaccent.map}


\mode<presentation>
{
  \usetheme{Warsaw} % or try Darmstadt, Madrid, Warsaw, Rochester, CambridgeUS, ...
  \usecolortheme{seahorse} % or try seahorse, beaver, crane, wolverine, ...
  \usefonttheme{serif}  % or try serif, structurebold, ...
  \setbeamertemplate{navigation symbols}{}
  \setbeamertemplate{caption}[numbered]
} 


%%%%%%%%%%%%%%%%%%%%%%%%%%%%
% itemize settings

\definecolor{mypink}{RGB}{255, 30, 80}
\definecolor{mydarkblue}{RGB}{60, 160, 255}
\definecolor{myblackblue}{RGB}{40, 40, 120}
\definecolor{myblue}{RGB}{240, 240, 255}
\definecolor{mygreen}{RGB}{0, 200, 0}
\definecolor{mygreen2}{RGB}{205, 255, 200}
\definecolor{mygray}{gray}{0.8}
% \definecolor{mydarkgray}{gray}{0.4}
\definecolor{mydarkgray}{RGB}{80, 80, 160}

\setbeamertemplate{itemize items}[default]

\setbeamertemplate{itemize item}{\color{myblackblue}$\blacksquare$}
\setbeamertemplate{itemize subitem}{\color{mydarkblue}$\blacktriangleright$}
\setbeamertemplate{itemize subsubitem}{\color{mygray}$\blacksquare$}

\setbeamercolor{palette quaternary}{fg=white,bg=mydarkgray}
\setbeamercolor{titlelike}{parent=palette quaternary}

\setbeamercolor{palette quaternary2}{fg=black,bg=myblue}
\setbeamercolor{frametitle}{parent=palette quaternary2}

\setbeamerfont{frametitle}{size=\Large,series=\scshape}
\setbeamerfont{framesubtitle}{size=\normalsize,series=\upshape}





%%%%%%%%%%%%%%%%%%%%%%%%%%%%
% block settings

\setbeamercolor{block title}{bg=red!30,fg=black}

\setbeamercolor*{block title example}{bg=mygreen!40!white,fg=black}

\setbeamercolor*{block body example}{fg= black, bg= mygreen2}


%%%%%%%%%%%%%%%%%%%%%%%%%%%%
% URL settings
\hypersetup{
    colorlinks=true,
    linkcolor=blue,
    filecolor=blue,      
    urlcolor=blue,
}

%%%%%%%%%%%%%%%%%%%%%%%%%%

\renewcommand{\familydefault}{\rmdefault}

\usepackage{amsmath}
\usepackage{mathtools}

\usepackage{subcaption}

\usepackage{qrcode}

\DeclareMathOperator*{\argmin}{arg\,min}
\newcommand{\bo}[1] {\mathbf{#1}}

\newcommand{\dx}[1] {\dot{\mathbf{#1}}}
\newcommand{\ma}[4] {\begin{bmatrix}
    #1 & #2 \\ #3 & #4
    \end{bmatrix}}
\newcommand{\myvec}[2] {\begin{bmatrix}
    #1 \\ #2
    \end{bmatrix}}
\newcommand{\myvecT}[2] {\begin{bmatrix}
    #1 & #2
    \end{bmatrix}}
    
    

\newcommand{\mydate}{Spring 2022}
\newcommand{\mygit}{\textcolor{blue}{\href{https://github.com/SergeiSa/Control-Theory-Slides-Spring-2022}{github.com/SergeiSa/Control-Theory-Slides-Spring-2022}}}


\newcommand{\bref}[2] {\textcolor{blue}{\href{#1}{#2}}}




%%%%%%%%%%%%%%%%%%%%%%%%%%%%
% code settings

\usepackage{listings}
\usepackage{color}
% \definecolor{mygreen}{rgb}{0,0.6,0}
% \definecolor{mygray}{rgb}{0.5,0.5,0.5}
\definecolor{mymauve}{rgb}{0.58,0,0.82}
\lstset{ 
  backgroundcolor=\color{white},   % choose the background color; you must add \usepackage{color} or \usepackage{xcolor}; should come as last argument
  basicstyle=\footnotesize,        % the size of the fonts that are used for the code
  breakatwhitespace=false,         % sets if automatic breaks should only happen at whitespace
  breaklines=true,                 % sets automatic line breaking
  captionpos=b,                    % sets the caption-position to bottom
  commentstyle=\color{mygreen},    % comment style
  deletekeywords={...},            % if you want to delete keywords from the given language
  escapeinside={\%*}{*)},          % if you want to add LaTeX within your code
  extendedchars=true,              % lets you use non-ASCII characters; for 8-bits encodings only, does not work with UTF-8
  firstnumber=0000,                % start line enumeration with line 0000
  frame=single,	                   % adds a frame around the code
  keepspaces=true,                 % keeps spaces in text, useful for keeping indentation of code (possibly needs columns=flexible)
  keywordstyle=\color{blue},       % keyword style
  language=Octave,                 % the language of the code
  morekeywords={*,...},            % if you want to add more keywords to the set
  numbers=left,                    % where to put the line-numbers; possible values are (none, left, right)
  numbersep=5pt,                   % how far the line-numbers are from the code
  numberstyle=\tiny\color{mygray}, % the style that is used for the line-numbers
  rulecolor=\color{black},         % if not set, the frame-color may be changed on line-breaks within not-black text (e.g. comments (green here))
  showspaces=false,                % show spaces everywhere adding particular underscores; it overrides 'showstringspaces'
  showstringspaces=false,          % underline spaces within strings only
  showtabs=false,                  % show tabs within strings adding particular underscores
  stepnumber=2,                    % the step between two line-numbers. If it's 1, each line will be numbered
  stringstyle=\color{mymauve},     % string literal style
  tabsize=2,	                   % sets default tabsize to 2 spaces
  title=\lstname                   % show the filename of files included with \lstinputlisting; also try caption instead of title
}


%%%%%%%%%%%%%%%%%%%%%%%%%%%%
% URL settings
\hypersetup{
    colorlinks=false,
    linkcolor=blue,
    filecolor=blue,      
    urlcolor=blue,
}

%%%%%%%%%%%%%%%%%%%%%%%%%%

%%%%%%%%%%%%%%%%%%%%%%%%%%%%
% tikz settings

\usepackage{tikz}
\tikzset{every picture/.style={line width=0.75pt}}


\title{Algebra in 3D: examples}
\subtitle{Math and modeling for high school}
\author{by Sergei Savin}
\centering
\date{Fall 2022}



\begin{document}
\maketitle


%\begin{frame}{Content}
%
%\begin{itemize}
%\item Motivation
%\item Ordinary differential equations
%    \begin{itemize}
%    \item 1st order
%    \item n-th order
%    \end{itemize}
%\item Linear differential equations
%    \begin{itemize}
%    \item 1st order
%    \item n-th order
%    \end{itemize}
%\item Changing n-th order ODE to a State-Space form
%\item State-Space to ODE
%\item Read more
%\end{itemize}
%
%\end{frame}



\begin{frame}{Problem statement}
	% \framesubtitle{Part 1}
	\begin{flushleft}
		
		Given a cube $ABCD A_1 B_1 C_1 D_1$ with a side length 9. Point $K \in BB_1$, with $||KB|| = 7$ (meaning the distance from $K$ to $B$ is 7). Plane $\alpha$ passes through $K$ and $C_1$, and is parallel to $BD_1$. $P$ is a point of intersection of $\alpha$ with $A_1 B_1$
		
		\bigskip
		
		\begin{itemize}
			\item 	Prove that $||A_1 P|| / ||PB_1|| = 2.5$.
			
			\item Find angle between $\alpha$ and $B B_1 C_1$.
		\end{itemize}
	
		
	\end{flushleft}
\end{frame}



\begin{frame}{Computational way of looking at the problem}
	% \framesubtitle{Part 1}
	\begin{flushleft}
		
		Our approach here can be straight-forward: we compute everything that can be computed exactly. 
		
		\bigskip
		
		Later we can discard useless steps, but it can be counter-productive to start discarding them before we know how we will arrive at the solution.
		
	\end{flushleft}
\end{frame}



\begin{frame}{Computational way of looking at the problem}
	% \framesubtitle{Part 1}
	\begin{flushleft}
		
		%B--A
		%C--D
		
		For example, assuming that $C_1 = [0, \ 0,  \ 0]$ (why? so $\alpha$ passes through the origin), and orienting the axes such that $D_1 = [9, \ 0,  \ 0]$, $B_1 = [0, \ 0,  \ 9]$, we get:
		
		\begin{align}
		A_1 = [9, \ 0,  \ 9] \\
		A = [9, \ 9,  \ 9] \\
		B = [0, \ 9,  \ 9] \\
		C = [0, \ 9,  \ 0] \\
		D = [9, \ 9,  \ 0]
		\end{align}
	
Do we need all these points? No. Is it trivial to find their coordinates? Yes. Is it easier to think about the problem when you know all their coordinates? Yes.	
		
	\end{flushleft}
\end{frame}



\begin{frame}{Computational way of looking at the problem}
	% \framesubtitle{Part 1}
	\begin{flushleft}
		
		Since  $K \in BB_1$ and $||KB|| = 7$, and $B = [0, \ 9,  \ 9]$, $B_1 = [0, \ 0,  \ 9]$, we can find coordinates of $K$:
		
		\begin{equation}
			K =  [0, \ 2,  \ 9]
		\end{equation}
		
		There are two ways to arrive there. 1) You know that $K$ is $7 / 9$ of the way between $B$ and $B_1$, so: $K = B + \frac{7}{9}(B_1-B) = \frac{2}{9}B + \frac{7}{9}B_1$. 
		
		\bigskip
		
		Or, 2) The direction from $B$ to $B_1$ is given as $\mathbf b = [0, \ -1,  \ 0]$, and we know that the distance is 7, so $K = B + 7\mathbf b$.
		
	\end{flushleft}
\end{frame}



\begin{frame}{Computational way of looking at the problem}
	% \framesubtitle{Part 1}
	\begin{flushleft}
		
		Plane $\alpha$ passes through $K$ and $C_1$, and is parallel to $BD_1$. The fact that it passes through $K$ and $C_1$, it means that a vector from  $K$ to $C_1$ is tangent to the plane. So is a vector from $B$ to $D_1$. Let us find those two vectors:
		
		\begin{equation}
			\bo{r}_{BD_1} = B-D_1 = [0, \ 9,  \ 9]-[9, \ 0,  \ 0] =  [-9, \ 9,  \ 9]
		\end{equation}
%
	\begin{equation}
	\bo{r}_{KC_1} = K-C_1 = [0, \ 2,  \ 9] - [0, \ 0,  \ 0] = [0, \ 2,  \ 9]
	\end{equation}
		
		\bigskip
		
		We can find the norm to the plane by taking a cross product of $\bo{r}_{BD_1}$ and $\bo{r}_{KC_1}$:
		
\begin{equation}
	\bo{n} = \bo{r}_{BD_1} \times \bo{r}_{KC_1} = \begin{bmatrix}
		81-18 \\
		0+81\\
		-18-0
	\end{bmatrix}
=	
	 \begin{bmatrix}
	 	  63 \\
	 	  81 \\
	 	-18
	 \end{bmatrix}
\end{equation}		
		
	\end{flushleft}
\end{frame}



\begin{frame}{Finding point P}
	% \framesubtitle{Part 1}
	\begin{flushleft}
		
		$P$ is a point of intersection of $\alpha$ with $A_1 B_1$. Let us find a vector from $A_1$ to $B_1$:
		%
		\begin{equation}
			\bo{r}_{A_1 B_1} = A_1 - B_1 = [9, \ 0,  \ 9] - [0, \ 0,  \ 9] = [9, \ 0,  \ 0]
		\end{equation}
	
	Let us denote coordinates of $P$ as $\mathbf p$ and coordinates of $B_1$ as $\mathbf b_1$. We know that $\mathbf p^\top \mathbf n = 0$ and $\mathbf p = \lambda \bo{r}_{A_1 B_1} + \mathbf b_1$. Therefore:
	
		\begin{equation}
	(\lambda \bo{r}_{A_1 B_1} + \mathbf b_1)^\top \mathbf n = 0
		\end{equation}	 
		\begin{equation}
			\lambda = -\frac{ \mathbf b_1^\top \mathbf n}{\bo{r}_{A_1 B_1}^\top \mathbf n}
		\end{equation}
%		
\begin{equation}
	\mathbf b_1^\top \mathbf n = 0 + 0 + 9 \cdot 2 = -162
\end{equation}		
%
\begin{equation}
\bo{r}_{A_1 B_1}^\top \mathbf n = -9 \cdot 11 + 0 + 0 =567
\end{equation}		
%
\begin{equation}
	\lambda = -\frac{ -162 }{567} = 2/7
\end{equation}

	
	\end{flushleft}
\end{frame}



\begin{frame}{Task 1}
	% \framesubtitle{Part 1}
	\begin{flushleft}
		
We found that $\lambda = 2/7$ and $\mathbf p = \lambda \bo{r}_{A_1 B_1} + \mathbf b_1$. Then:

\begin{equation}
	\mathbf p =  2/7  [9, \ 0,  \ 0] +  [0, \ 0,  \ 9] = [18/7, \ 0,  \ 9] 
\end{equation}
		
		\bigskip
		
		We can find $|| A_1 P ||$ and $|| B_1 P ||$; These vectors are $A_1 P = [45/7, \ 0,  \ 0] $ and $B_1 P = [18/7, \ 0,  \ 0] $. With that we know:
		
		\begin{align}
			||A_1 P|| = 45/7 \\
			||B_1 P|| = 18/7 \\
			||A_1 P|| / ||B_1 P|| = 45 \ 18 = 5/2
		\end{align}
		
	\end{flushleft}
\end{frame}



\begin{frame}{Task 2}
	% \framesubtitle{Part 1}
	\begin{flushleft}
		
		Find angle between $\alpha$ and $B B_1 C_1$. This is the same as angle between the normals to these planes. First we find the norm to the plane $B B_1 C_1$:
		
		
\begin{align}
	B - B_1 = [0, \ 9,  \ 0] \\
	B_1 - C_1 = [0, \ 0,  \ 9]  \\
	\bo{m} = (B - B_1) \times (B_1 - C_1) = [81, \ 0,  \ 0]
\end{align}		
		
		With that we know normals to both $\alpha$ and $B B_1 C_1$, and can find the angle between them using dot product.
		
	\end{flushleft}
\end{frame}


\begin{frame}{Task 2}
	% \framesubtitle{Part 1}
	\begin{flushleft}
		
		We know that:
		
		\begin{equation}
			\bo{m} \cdot \bo{n} = || \bo{n}|| \ || \bo{m}|| \text{cos}(\varphi),
		\end{equation}
	
	where $\varphi$ is the angle we seek. We need to find $\bo{m} \cdot \bo{n}$, as well as $|| \bo{m}||$ and $|| \bo{n}||$:
		
		\begin{align}
			\bo{m} \cdot \bo{n} = 63 \cdot 81 + 0 + 0 = 5103 \\
			|| \bo{n}|| = \sqrt{10854} \\
			|| \bo{m}|| = 81 \\
			\text{cos}(\varphi) = \frac{\bo{m} \cdot \bo{n} }{|| \bo{n}|| \ || \bo{m}||} = 5103 / (81 \sqrt{10854}) \approx 0.6047 \\
			\varphi = 52.79^{\circ}
		\end{align}		
		
	\end{flushleft}
\end{frame}







\begin{frame}[fragile]{The code in Matlab for this problem is:}
	% \framesubtitle{Part 1}
	\begin{flushleft}
		
		\begin{lstlisting}[language=Matlab]
			dx = [9;0;0]; dy = [0;9;0]; dz = [0;0;9];
			C1 = sym([0;0;0]);
			D1 = C1 + dx; A1 = C1 + dx + dz; B1 = C1 + dz;
			B = C1 + dy + dz;
			
			K = (2/9) * B + (7/9) * B1;
			BD1 = B - D1;
			KC1 = K - C1;
			n = cross(BD1, KC1);
			lambda = -dot(B1, n) / dot(dx, n);
			P = lambda*dx + B1;
			A1P = A1 - P;
			B1P = B1 - P;
			disp( norm(A1P) / norm(B1P) ) %task 1
			
			m = cross(dy, dz); 
			cos_phi = dot(n, m) / (norm(n) * norm(m));
			%task 2:
			phi = round(double(acos(cos_phi))*180/pi, 3) 
		\end{lstlisting}
		
		
	\end{flushleft}
\end{frame}


\begin{frame}{Homework}
	% \framesubtitle{Part 1}
	\begin{flushleft}
		
		Consider a cube $ABCD A_1 B_1 C_1 D_1$ with a side length 3. Point $S_1 \in DB_1$, with $||DS_1|| = 1$ (meaning the distance from $D$ to $S_1$ is 1). Point $S_2 \in AB_1$, with $||AS_2|| = 2$ (meaning the distance from $A$ to $S_2$ is 2). Plane $\alpha_1$ passes through $S_1$, $S_2$ and $D$. Plane $\alpha_2$ is orthogonal to $C_1S_1$ through $C$. Point $P$ is an intersection between the plane $\alpha_1$ and the line passing through points $D_1$, $B$.
		
		\bigskip
		
		\begin{itemize}
			\item Find distance between $P$ and $A$.
			\item Find angle between $\alpha_1$ and $\alpha_2$.
			\item Find distance between $\alpha_1$ and all vertices of the cube.
			\item Prove that $P$ lies on $B_1 S_1$.
		\end{itemize}
		
		
	\end{flushleft}
\end{frame}




\begin{frame}{Thank you!}
\centerline{Lecture slides are available via Moodle.}
\bigskip
\centerline{You can help improve these slides at:}
\centerline{\mygit}
\bigskip
\centerline{Check Moodle for additional links, videos, textbook suggestions.}
\bigskip

\centerline{\textcolor{black}{\qrcode[height=1.6in]{https://github.com/SergeiSa/Extra-math-for-high-school}}}

\end{frame}

\end{document}
