\documentclass{beamer}

\pdfmapfile{+sansmathaccent.map}


\mode<presentation>
{
  \usetheme{Warsaw} % or try Darmstadt, Madrid, Warsaw, Rochester, CambridgeUS, ...
  \usecolortheme{seahorse} % or try seahorse, beaver, crane, wolverine, ...
  \usefonttheme{serif}  % or try serif, structurebold, ...
  \setbeamertemplate{navigation symbols}{}
  \setbeamertemplate{caption}[numbered]
} 


%%%%%%%%%%%%%%%%%%%%%%%%%%%%
% itemize settings

\definecolor{mypink}{RGB}{255, 30, 80}
\definecolor{mydarkblue}{RGB}{60, 160, 255}
\definecolor{myblackblue}{RGB}{40, 40, 120}
\definecolor{myblue}{RGB}{240, 240, 255}
\definecolor{mygreen}{RGB}{0, 200, 0}
\definecolor{mygreen2}{RGB}{205, 255, 200}
\definecolor{mygray}{gray}{0.8}
% \definecolor{mydarkgray}{gray}{0.4}
\definecolor{mydarkgray}{RGB}{80, 80, 160}

\setbeamertemplate{itemize items}[default]

\setbeamertemplate{itemize item}{\color{myblackblue}$\blacksquare$}
\setbeamertemplate{itemize subitem}{\color{mydarkblue}$\blacktriangleright$}
\setbeamertemplate{itemize subsubitem}{\color{mygray}$\blacksquare$}

\setbeamercolor{palette quaternary}{fg=white,bg=mydarkgray}
\setbeamercolor{titlelike}{parent=palette quaternary}

\setbeamercolor{palette quaternary2}{fg=black,bg=myblue}
\setbeamercolor{frametitle}{parent=palette quaternary2}

\setbeamerfont{frametitle}{size=\Large,series=\scshape}
\setbeamerfont{framesubtitle}{size=\normalsize,series=\upshape}





%%%%%%%%%%%%%%%%%%%%%%%%%%%%
% block settings

\setbeamercolor{block title}{bg=red!30,fg=black}

\setbeamercolor*{block title example}{bg=mygreen!40!white,fg=black}

\setbeamercolor*{block body example}{fg= black, bg= mygreen2}


%%%%%%%%%%%%%%%%%%%%%%%%%%%%
% URL settings
\hypersetup{
    colorlinks=true,
    linkcolor=blue,
    filecolor=blue,      
    urlcolor=blue,
}

%%%%%%%%%%%%%%%%%%%%%%%%%%

\renewcommand{\familydefault}{\rmdefault}

\usepackage{amsmath}
\usepackage{mathtools}

\usepackage{subcaption}

\usepackage{qrcode}

\DeclareMathOperator*{\argmin}{arg\,min}
\newcommand{\bo}[1] {\mathbf{#1}}

\newcommand{\dx}[1] {\dot{\mathbf{#1}}}
\newcommand{\ma}[4] {\begin{bmatrix}
    #1 & #2 \\ #3 & #4
    \end{bmatrix}}
\newcommand{\myvec}[2] {\begin{bmatrix}
    #1 \\ #2
    \end{bmatrix}}
\newcommand{\myvecT}[2] {\begin{bmatrix}
    #1 & #2
    \end{bmatrix}}
    
    

\newcommand{\mydate}{Spring 2022}
\newcommand{\mygit}{\textcolor{blue}{\href{https://github.com/SergeiSa/Control-Theory-Slides-Spring-2022}{github.com/SergeiSa/Control-Theory-Slides-Spring-2022}}}


\newcommand{\bref}[2] {\textcolor{blue}{\href{#1}{#2}}}




%%%%%%%%%%%%%%%%%%%%%%%%%%%%
% code settings

\usepackage{listings}
\usepackage{color}
% \definecolor{mygreen}{rgb}{0,0.6,0}
% \definecolor{mygray}{rgb}{0.5,0.5,0.5}
\definecolor{mymauve}{rgb}{0.58,0,0.82}
\lstset{ 
  backgroundcolor=\color{white},   % choose the background color; you must add \usepackage{color} or \usepackage{xcolor}; should come as last argument
  basicstyle=\footnotesize,        % the size of the fonts that are used for the code
  breakatwhitespace=false,         % sets if automatic breaks should only happen at whitespace
  breaklines=true,                 % sets automatic line breaking
  captionpos=b,                    % sets the caption-position to bottom
  commentstyle=\color{mygreen},    % comment style
  deletekeywords={...},            % if you want to delete keywords from the given language
  escapeinside={\%*}{*)},          % if you want to add LaTeX within your code
  extendedchars=true,              % lets you use non-ASCII characters; for 8-bits encodings only, does not work with UTF-8
  firstnumber=0000,                % start line enumeration with line 0000
  frame=single,	                   % adds a frame around the code
  keepspaces=true,                 % keeps spaces in text, useful for keeping indentation of code (possibly needs columns=flexible)
  keywordstyle=\color{blue},       % keyword style
  language=Octave,                 % the language of the code
  morekeywords={*,...},            % if you want to add more keywords to the set
  numbers=left,                    % where to put the line-numbers; possible values are (none, left, right)
  numbersep=5pt,                   % how far the line-numbers are from the code
  numberstyle=\tiny\color{mygray}, % the style that is used for the line-numbers
  rulecolor=\color{black},         % if not set, the frame-color may be changed on line-breaks within not-black text (e.g. comments (green here))
  showspaces=false,                % show spaces everywhere adding particular underscores; it overrides 'showstringspaces'
  showstringspaces=false,          % underline spaces within strings only
  showtabs=false,                  % show tabs within strings adding particular underscores
  stepnumber=2,                    % the step between two line-numbers. If it's 1, each line will be numbered
  stringstyle=\color{mymauve},     % string literal style
  tabsize=2,	                   % sets default tabsize to 2 spaces
  title=\lstname                   % show the filename of files included with \lstinputlisting; also try caption instead of title
}


%%%%%%%%%%%%%%%%%%%%%%%%%%%%
% URL settings
\hypersetup{
    colorlinks=false,
    linkcolor=blue,
    filecolor=blue,      
    urlcolor=blue,
}

%%%%%%%%%%%%%%%%%%%%%%%%%%

%%%%%%%%%%%%%%%%%%%%%%%%%%%%
% tikz settings

\usepackage{tikz}
\tikzset{every picture/.style={line width=0.75pt}}


\title{Linear equations, matrices and vectors}
\subtitle{Math and modelling for high school, Lecture 1}
\author{by Sergei Savin}
\centering
\date{Fall 2022}



\begin{document}
\maketitle


%\begin{frame}{Content}
%
%\begin{itemize}
%\item Motivation
%\item Ordinary differential equations
%    \begin{itemize}
%    \item 1st order
%    \item n-th order
%    \end{itemize}
%\item Linear differential equations
%    \begin{itemize}
%    \item 1st order
%    \item n-th order
%    \end{itemize}
%\item Changing n-th order ODE to a State-Space form
%\item State-Space to ODE
%\item Read more
%\end{itemize}
%
%\end{frame}



\begin{frame}{What is this course about?}
% \framesubtitle{O}
\begin{flushleft}

In this course we try to do the following:

\begin{itemize}
	\item Teach you basics of university-level linear algebra, basic introduction to calculus and differential equations and simulation;
	\item Focus on visual aspect of math: computer graphics, graphs, animation. We do it to use your strength - programming, to help you understand inherently abstract concepts. 
	\item We try to keep the topics coherent and compact, avoiding topics that require deeper algebraic or calculus theory, which forces us to skip many interesting things - which you will surely learn in the university.
\end{itemize}

\end{flushleft}
\end{frame}






\begin{frame}{Recap: Matrices}
	% \framesubtitle{Part 1}
	\begin{flushleft}
		
		The following is a 3 by 3 matrix:
		
		\begin{equation}
			\bo{A} =
			\begin{bmatrix}
			a_{11} & a_{12} & a_{13} \\
			a_{21} & a_{22} & a_{23} \\
			a_{31} & a_{32} & a_{33}
			\end{bmatrix}
		\end{equation}
		
		\bigskip
		Examples of 3 by 3 matrices:
		
		\begin{equation}
			\bo{A} =
			\begin{bmatrix}
				1 & 0 & 0 \\
				2 & 6 & -5 \\
				0 & -3 & 3
			\end{bmatrix},
		\ \ 
		\bo{A} =
		\begin{bmatrix}
			10 & 2 & 0 \\
			-4 & 5 & 0 \\
			-4 & 5 & 2
		\end{bmatrix},  \ \ \text{etc}
		\end{equation}
	
		
	\end{flushleft}
\end{frame}



\begin{frame}{Recap: Vectors}
	% \framesubtitle{Part 1}
	\begin{flushleft}
		
		The following is a 3-dimensional vector:
		
		\begin{equation}
			\bo{v} =
			\begin{bmatrix}
				v_{1}  \\
				v_{2}  \\
				v_{3}
			\end{bmatrix}
		\end{equation}
		
		\bigskip
		Examples 3-dimensional vectors:
		
		\begin{equation}
			\bo{v} =
			\begin{bmatrix}
				5 \\
				5 \\
				0
			\end{bmatrix},
			\ \ 
			\bo{v} =
			\begin{bmatrix}
				4 \\
				-3 \\
				-1
			\end{bmatrix},  \ \ \text{etc}
		\end{equation}
		
	\end{flushleft}
\end{frame}


\begin{frame}{Recap: Norms}
	% \framesubtitle{Part 1}
	\begin{flushleft}
		
		Given a vector: 
			$\bo{v} =
			\begin{bmatrix}
				v_{1}  \\
				v_{2}  \\
				v_{3}
			\end{bmatrix}$, we can find its \emph{norm}:
		
		\bigskip
		
		\begin{equation}
			||\bo{v}|| =
			\sqrt{v_{1}^2 + v_{2}^2+ v_{3}^2}
		\end{equation}
	
	\bigskip
	
	You can think of it as \emph{length} of a vector.
		
	\end{flushleft}
\end{frame}



\begin{frame}{Recap: Matrix-vector product}
	% \framesubtitle{Part 1}
	\begin{flushleft}
		
		We can find matrix-vector product:
		
		\begin{equation}
			\begin{bmatrix}
				a_{11} & a_{12} & a_{13} \\
				a_{21} & a_{22} & a_{23} \\
				a_{31} & a_{32} & a_{33}
			\end{bmatrix}
			\begin{bmatrix}
				v_{1}  \\
				v_{2}  \\
				v_{3}
			\end{bmatrix} =
			\begin{bmatrix}
				a_{11} v_{1} + a_{12} v_{2} + a_{13}v_{3} \\
				a_{21} v_{1} + a_{22} v_{2} + a_{23}v_{3} \\
				a_{31} v_{1} + a_{32} v_{2} + a_{33}v_{3}
			\end{bmatrix}
		\end{equation}
	
		\bigskip
		
		An example of a matrix-vector product:
		
		\begin{equation}
			\begin{bmatrix}
				1 & 0 & 0 \\
				2 & 6 & -5 \\
				0 & -3 & 3
			\end{bmatrix}
			\begin{bmatrix}
				1  \\
				-2  \\
				0
			\end{bmatrix} =
			\begin{bmatrix}
				1 \\
				-10 \\
				6
			\end{bmatrix}
		\end{equation}
		
	\end{flushleft}
\end{frame}





\begin{frame}{Matrices and vectors - recap}
	% \framesubtitle{Part 1}
	\begin{flushleft}
		
		Matrices don't need to be square. For the matrix-vector product to work, the matrix needs to have as many columns as there are elements in the vector. 
		
		\begin{equation}
			\begin{bmatrix}
				a_{11} & a_{12} & a_{13} \\
				a_{21} & a_{22} & a_{23} 
			\end{bmatrix}
			\begin{bmatrix}
				v_{1}  \\
				v_{2}  \\
				v_{3}
			\end{bmatrix} =
			\begin{bmatrix}
				a_{11} v_{1} + a_{12} v_{2} + a_{13}v_{3} \\
				a_{21} v_{1} + a_{22} v_{2} + a_{23}v_{3} 
			\end{bmatrix}
		\end{equation}
		
		\bigskip
		
		An example of a matrix-vector product:
		
		\begin{equation}
			\begin{bmatrix}
				2 & 2 & -1 \\
				0 & 5 & 7 
			\end{bmatrix}
			\begin{bmatrix}
				0  \\
				2  \\
				-2
			\end{bmatrix} =
			\begin{bmatrix}
				6 \\
				-4
			\end{bmatrix}
		\end{equation}
		
	\end{flushleft}
\end{frame}



\begin{frame}{Read more}
	% \framesubtitle{Part 1}
	\begin{flushleft}
		
		There are many resources on the topic:
		
\begin{itemize}
	\item \bref{https://mathinsight.org/matrix_introduction}{mathinsight.org/matrix\_introduction}
	
	\item \bref{https://mathinsight.org/matrix_vector_multiplication}{mathinsight.org/matrix\_vector\_multiplication}
	
	\item \bref{https://mathworld.wolfram.com/L2-Norm.html}{mathworld.wolfram.com/L2-Norm.html}
	
	\item etc.
\end{itemize}		
		
	\end{flushleft}
\end{frame}


\begin{frame}{Matrices and vectors in Python}
	% \framesubtitle{Part 1}
	\begin{flushleft}
		
		In Python we can use Numpy package to create matrices and vectors:
		
\begin{itemize}
	\item matrix: \texttt{A = np.array([[1, -4, 5], [3, 1, -2], [2, -6, -9]])}
	
	\item vector: \texttt{v = np.array([1, 1, -2])}
	
\end{itemize}		

\bigskip
	
	Matrix-vector product is done as \texttt{y = A@v}.
		
	\end{flushleft}
\end{frame}





\begin{frame}{Systems of linear equations}
% \framesubtitle{Part 1}
\begin{flushleft}

Consider the system of equations:

\begin{equation}
	\begin{cases}
		x_1   + 2 x_2 = 10 \\
		2 x_1 + 2 x_2 = 0 
	\end{cases}
\end{equation}

This is the same as:

\begin{equation}
	\begin{bmatrix}
		1   & 2  \\
		2   & 2  
	\end{bmatrix}
	\begin{bmatrix}
		x_1  \\
		x_2 
	\end{bmatrix}
	=
	\begin{bmatrix}
		10  \\
		0 
	\end{bmatrix}
\end{equation}

\bigskip

General form for this kind of problem is:


\begin{equation}
	\mathbf{A} \mathbf{x} = \mathbf{b}
\end{equation}

%In this course we will solve it with $\texttt{x = np.linalg.solve(A, b)}$ fucntionality of numpy. 


\end{flushleft}
\end{frame}





\begin{frame}[fragile]{Systems of linear equations}
	% \framesubtitle{Part 1}
	\begin{flushleft}
		
		In this course we will solve it with $\texttt{x = np.linalg.solve(A, b)}$ functionality of numpy:
		
		\bigskip
		
		\begin{lstlisting}[language=Python]
			import numpy as np
			A = np.array([[1, 2], [2, 2]])
			b = np.array([10, 0])
			x = np.linalg.solve(A, b)
			print("solution:")
			print(x)
			print("residual:")
			print(A@x - b)
		\end{lstlisting}
		
		
	\end{flushleft}
\end{frame}




\begin{frame}{Systems of linear equations: Other dimensions}
	% \framesubtitle{Part 1}
	\begin{flushleft}
		
		Dimensions do not matter, we can deal with equations with 2, 3 and more variables the same way:
		
		\begin{equation}
			\begin{cases}
				x_1   - 4 x_2 + 5 x_3= 1 \\
				3 x_1 +   x_2 - 2 x_3= 1 \\ 
				2 x_1 - 6 x_2 - 9 x_3= -2 
			\end{cases}
		\end{equation}
		
		This is the same as
		
		\begin{equation}
			\begin{bmatrix}
				1   & -4 &  5  \\
				3   &  1 & -2  \\  
				2   & -6 & -9   
			\end{bmatrix}
			\begin{bmatrix}
				x_1  \\
				x_2  \\
				x_3 
			\end{bmatrix}
			=
			\begin{bmatrix}
				1  \\
				1   \\
				-2 
			\end{bmatrix}
		\end{equation}
		
		Again, it is $\mathbf{A} \mathbf{x} = \mathbf{b}$ and we solve it with $\texttt{x = np.linalg.solve(A, b)}$. 
	\end{flushleft}
\end{frame}




\begin{frame}{Dimensions: notation}
	% \framesubtitle{Part 1}
	\begin{flushleft}
		
		When we want to note the dimension of a vector, we use the following notation:
		
		\begin{equation}
			\mathbf{b} \in \mathbb{R}^n
		\end{equation}
		
		where $\mathbb{R}$ means space of real numbers and $n$ is the dimensions.
		
		\bigskip
		
		For matrices it is similar:
		
		\begin{equation}
			\mathbf{A} \in \mathbb{R}^{n, n}
		\end{equation}
		
		Why use this strange $\mathbb{R}$? Because later you might see something like $\mathbf{b} \in \mathbb{C}^n$, meaning it is a complex vector, or $\mathbf{b} \in \mathbb{N}^n$ meaning it is a vector of integers. 
		
	\end{flushleft}
\end{frame}




\begin{frame}{Degenerate eqs.: too many variables}
	% \framesubtitle{Part 1}
	\begin{flushleft}
	
Let us solve a degenerate system of equations:

\begin{equation}
	x_1   + 2 x_2 = 10
\end{equation}

This is the same as

\begin{equation}
	\begin{bmatrix}
		1   & 2
	\end{bmatrix}
	\begin{bmatrix}
		x_1  \\
		x_2 
	\end{bmatrix}
	=
	10
\end{equation}

General form for this kind of problem is still:

\begin{equation}
	\mathbf{A} \mathbf{x} = \mathbf{b}
\end{equation}

We can solve it with \texttt{x, \_, \_, \_ = np.linalg.lstsq(A, b, rcond=None)} fucntionality of numpy. 		
		
	\end{flushleft}
\end{frame}



\begin{frame}{Degenerate eqs.: too many variables}
	% \framesubtitle{Part 1}
	\begin{flushleft}
		
		But what does it even mean to solve $x_1   + 2 x_2 = 10$? Let us try to solve it by hand.
		
		\bigskip
		
		We know that $x_1  =10 -  2 x_2$. And we can pick any $x_2$. For example, $x_2 = 1$, then $x_1  =10 -  2 = 8$. Or, $x_2 = 0$, then $x_1  =10$. Or, $x_2 = 4$, then $x_1  =10 - 8 = 2$. So, all the following $\bo{x}$ are solutions:
		
		\begin{equation}
	\bo{x} =
	\begin{bmatrix}
		8 \\
		1
	\end{bmatrix},
	\ \ 
	\bo{x} =
	\begin{bmatrix}
		10 \\
		0
	\end{bmatrix},
\ \ 
	\bo{x} =
	\begin{bmatrix}
	2 \\
	4
	\end{bmatrix}.
\end{equation}		
		
The numpy function \texttt{lstsq} will pick \emph{smallest norm solution} out of all possible ones. In this case, it is, $\bo{x} =
\begin{bmatrix}
	2 \\
	4
\end{bmatrix}$.
		
	\end{flushleft}
\end{frame}





\begin{frame}{Too many equations}
	% \framesubtitle{Part 1}
	\begin{flushleft}

Let us solve a degenerate system of equations:

\begin{equation}
	\begin{cases}
		x_1   + 2 x_2 = 2 \\
		3 x_1   - 5 x_2 = 0  \\
		2 x_1   - 4 x_2 = 5
	\end{cases}
\end{equation}

This is the same as

\begin{equation}
	\begin{bmatrix}
		1   &  2 \\
		3   & -5 \\ 
		2   & -4
	\end{bmatrix}
	\begin{bmatrix}
		x_1  \\
		x_2 
	\end{bmatrix}
	=
	\begin{bmatrix}
		2 \\ 0 \\ 5
	\end{bmatrix}
\end{equation}

General form for this kind of problem is still:

\begin{equation}
	\mathbf{A} \mathbf{x} = \mathbf{b}
\end{equation}

Again, we just use \texttt{x, \_, \_, \_ = np.linalg.lstsq(A, b, rcond=None)}. 		
		
	\end{flushleft}
\end{frame}





\begin{frame}[fragile]{Degenerate system of equations}
	% \framesubtitle{Part 1}
	\begin{flushleft}
		
\begin{lstlisting}[language=Python]
A = np.array([[1, 2], [3, -5], [2, -4]])
b = np.array([[2], [0], [5]])
x, _, _, _ = np.linalg.lstsq(A, b, rcond=None)
e = A@x - b;
print("solution: " + str(x.T));
print("residual: " + str(e.T));
\end{lstlisting}		
%
The output for this example is:		
%
\begin{lstlisting}[language=Python]
solution: [[1.07936508 0.14814815]]
residual: [[-0.62433862  2.4973545  -3.43386243]]
\end{lstlisting}			
%
Notice the residual is \textbf{not zero}. Note that the function \texttt{lstsq} will pick solution with \emph{smallest-norm residual}.
		
	\end{flushleft}
\end{frame}



\begin{frame}{Degenerate square system}
	% \framesubtitle{Part 1}
	\begin{flushleft}

The following system is square (same number of variable and equations) but you still cannot solve it normally:

\begin{equation}
	\begin{cases}
		x_1   - 2 x_2 = 0 \\
		3 x_1   - 6 x_2 = 2 
	\end{cases}
\end{equation}

This is the same as:

\begin{equation}
	\begin{bmatrix}
		1   & -2 \\
		3   & -6 
	\end{bmatrix}
	\begin{bmatrix}
		x_1  \\
		x_2 
	\end{bmatrix}
	=
	\begin{bmatrix}
		0 \\ 2
	\end{bmatrix}
\end{equation}

And it does not have a zero-residual solution.		
		
	\end{flushleft}
\end{frame}




\begin{frame}{How do you tell if a system can be solved?}
	% \framesubtitle{Part 1}
	\begin{flushleft}
		
		The test is - to find if the matrix \textit{determinant} is not zero. 	
		For 2 by 2 matrices, the determinant is found as follows:
		
		\begin{equation}
			\mathbf A =
			\begin{bmatrix}
				a   & b  \\
				c   & d  
			\end{bmatrix}
		\end{equation}
		
		\begin{equation}
			\text{det} (\mathbf A) =
			ad - bc
		\end{equation}
		
		\bigskip
		
		For anything other than 2 by 2 we find determinant as $\texttt{d = np.linalg.det(A)}$ fucntionality of numpy. 
		
		\bigskip
		
		If a matrix has 0 determinant we call it \textit{degenerate}. If the matrix has non-zero determinant, we call it \textit{full-rank}.
		
	\end{flushleft}
\end{frame}



\begin{frame}{Linear combination}
	% \framesubtitle{Part 1}
	\begin{flushleft}
		
	Given a few vectors $\mathbf v_1$, $\mathbf v_2$, ..., $\mathbf v_n \in \mathbb{R}^n$, their linear combination is defined as:
	
	\begin{equation}
		\mathbf w = \alpha_1 \mathbf v_1 + \alpha_2 \mathbf v_2 + ... + \alpha_n \mathbf v_n
	\end{equation}
	
	where $\alpha_i \in \mathbb{R}$ are linear coefficients.
	
	\end{flushleft}
\end{frame}




\begin{frame}{Linear combination}
	% \framesubtitle{Part 1}
	\begin{flushleft}
		
		Example: Given $\mathbf v_1$ and $\mathbf v_2$:
		
		\begin{equation}
			\mathbf v_1 = 
			\begin{bmatrix} 
				1 \\ 3
			\end{bmatrix}, \ \ 
			\mathbf v_2 = 
			\begin{bmatrix} 
				0 \\ 1
			\end{bmatrix}
		\end{equation}
		
		Then examples of their linear combinations are:
		
		\begin{equation}
			\mathbf w = 2 \mathbf v_1 - \mathbf v_2 = 
			2
			\begin{bmatrix} 
				1 \\ 3
			\end{bmatrix}
			-
			\begin{bmatrix} 
				0 \\ 1
			\end{bmatrix}
			=
			\begin{bmatrix} 
				2 \\ 5
			\end{bmatrix}
		\end{equation}
		
		\begin{equation}
			\mathbf w = - 3\mathbf v_1 + 2\mathbf v_2 = 
			-3
			\begin{bmatrix} 
				1 \\ 3
			\end{bmatrix}
			+2
			\begin{bmatrix} 
				0 \\ 1
			\end{bmatrix}
			=
			\begin{bmatrix} 
				-3 \\ -7
			\end{bmatrix}
		\end{equation}
		
		
		\begin{equation}
			\mathbf w = 5\mathbf v_2 = 
			5
			\begin{bmatrix} 
				0 \\ 1
			\end{bmatrix}
			=
			\begin{bmatrix} 
				0 \\ 5
			\end{bmatrix}
		\end{equation}
		
		If a set of vectors contain none that is a linear combination of the others - we call them \textit{linearly independant}.	
		
	\end{flushleft}
\end{frame}




\begin{frame}{Linear combination and degenerate matrix}
	% \framesubtitle{Part 1}
	\begin{flushleft}
	
	If one of your matrix's columns is a linear combination of the other - it will be degenerate. Lets look for an example. Consider the matrix:
	
	\begin{equation}
		\mathbf A =
		\begin{bmatrix}
			a_{11}   & a_{12}   \\
			a_{21}   & a_{22}   
		\end{bmatrix}
	\end{equation}
	
	Its columns are $\mathbf a_1$ and $\mathbf a_2$:
	
	\begin{equation}
		\mathbf a_1 =
		\begin{bmatrix}
			a_{11}      \\
			a_{21}     
		\end{bmatrix}, \ \ 
		\mathbf a_2 =
		\begin{bmatrix}
			a_{21}      \\
			a_{22}     
		\end{bmatrix}
	\end{equation}
	
	If, for instance, $\mathbf a_2 = 2 \mathbf a_1$, the matrix $\mathbf A$ will be degenerate.
		
	\end{flushleft}
\end{frame}



\begin{frame}{Linear combination and degenerate matrix}
	% \framesubtitle{Part 1}
	\begin{flushleft}
		
An example of a matrix with linearly dependent columns:
		
		\begin{equation}
			\mathbf a_1 =
			\begin{bmatrix}
				2      \\
				-3     
			\end{bmatrix}, 
		\ \ 
		\mathbf a_2 = -\mathbf a_1 =
		\begin{bmatrix}
			-2      \\
			3     
		\end{bmatrix}
		\end{equation}
		
		
		Then matrix $\mathbf A =
		\begin{bmatrix}
			2    & -2   \\
			-3   & 3   
		\end{bmatrix}$ is degenerate, and $\text{det} (\mathbf A) = 0$. Meaning, you can't solve a system like this one:
		
		
		\begin{equation}
			\begin{cases}
				2 x_1   - 2 x_2 = 1 \\
				-3 x_1   + 3 x_2 = 0 
			\end{cases}
		\end{equation}
		
		Really though? Let us check. $-3 x_1   + 3 x_2 = 0$, therefore $x_1 = x_2$, and since $ 2 x_1   - 2 x_2 = 1$ we get $ 2 x_1   - 2 x_1 = 1$ and so $ 0 = 1$, which is clearly incorrect.
		
	\end{flushleft}
\end{frame}




\begin{frame}{Read more}

\begin{itemize}
\item \bref{https://mathinsight.org/matrices_determinants_multivariable_calculus}{mathinsight.org/matrices\_determinants}

\item \bref{https://en.wikipedia.org/wiki/Determinant}{en.wikipedia.org/wiki/Determinant}

\end{itemize}

\end{frame}



\begin{frame}{Thank you!}
\centerline{Lecture slides are available via Moodle.}
\bigskip
\centerline{You can help improve these slides at:}
\centerline{\mygit}
\bigskip
\centerline{Check Moodle for additional links, videos, textbook suggestions.}
\bigskip

\centerline{\textcolor{black}{\qrcode[height=1.6in]{https://github.com/SergeiSa/Control-Theory-Slides-Spring-2022}}}

\end{frame}

\end{document}
